\section{Code Technical Debt}
I sistemi di intelligenza artificiale sono composti per la maggior parte da codice di supporto alla trasformazione di un modello in un prodotto, come già definito in \ref{sec:se4ai}. Come nei sistemi tradizionali, il technical debt quindi può emergere in questi sistemi per causa dell'introduzione di cattive pratiche del software.
Bogner et al. \cite{bogner2021characterizing} hanno condotto uno studio per analizzare la diffusione dei code smell presenti nei sistemi di intelligenza artificiale. I risultati di questo studio hanno fatto emergere la presenza di \textit{dead experimental code paths} all'interno del codice di produzione. Questa particolare tipologia di smell prevede l'inserimento di sezioni di codice come rami condizionali al fine di poter sperimentare le nuove funzionalità. L'accumulo di codice sperimentale all'interno del sistema di produzione può creare un crescente technical debt che affligge la complessità del sistema e la sua manutenibilità.

Jebnoun et al. \cite{Jebnoun}, attraverso un'analisi comparativa tra progetti di deep learning e progetti tradizionali, hanno riscontrato un'alta frequenza di code smell relativi alla complessità e la lunghezza delle espressioni, arrivando alla conclusione che il codice relativo ai modelli di deep learning includono espressioni più lunghe e più complesse.

Gesi et al. \cite{GesiCSDeepL} hanno inoltre analizzato la diffusione e la severità di code smell all'interno dei sistemi di deep learning in Python, investigando sulla percezione degli sviluppatori. I risultati mostrano un alta frequenza dei seguenti smell:
\begin{itemize}
    \item Scattered Use of ML Libraries: Indica l'utilizzo non sistematico delle librerie di machine learning. All'occorrenza di una modifica sulla determinata libreria, riportare gli aggiornamenti e le modifiche diventa un'operazione onerosa da dover effettuare in diversi punti del sistema.
    \item Unwanted Debugging Code: Indica la presenza di frammenti di codice non più utilizzati all'interno del sistema, aumentando senza alcun beneficio la dimensione e la complessità del sistema.
    \item Deep God File: E' la rappresentazione della God Class nei sistemi di deep learning (come descritta in sezione \ref{sec:code_debt}). Questa componente ingloba diverse funzionalità relative alle fasi del ciclo di vita di un modello di deep learning. 
    \item Jumbled Model Architecture: Le parti dell'architettura del sistema di deep learning sono inglobate aumentando la difficoltà di comprensione e di manutenzione.
\end{itemize}

Queste diverse definizioni della tassonomia di technical debt per sistemi AI permettono di poter analizzare la definizione da diversi punti di vista e con diverse granularità.
Tuttavia, la tassonomia definita dai precedenti autori ha bisogno di maggiori approfondimenti, che permettono di identificare la natura e la causa del technical debt nei sistemi AI.
Naturalmente, essendo le categorie stesse di techincal debt ancora soggette a definizione e raffinamento, è evidente la mancanza di tool automatici a supporto della loro identificazione all'interno di progetti AI-based.
Attualmente, le analisi condotte e presenti nello stato dell'arte prevedono l'applicazione nei sistemi AI della tassonomia dei technical debt presenti nei sistemi tradizionali, senza andare nel dettaglio in possibili istanze che causano technical debt specifiche per l'intelligenza artificiale.
Inoltre molti studi risultano essere complementari al lavoro di tesi, in quanto nello stato dell'arte è presente un maggiore livello di dettaglio su aspetti del sistema di intelligenza artificiale riguardante i dati e il modello. Il presente lavoro di tesi, quindi, contribuisce invece ad aumentare il livello di dettaglio di smell specifici per AI sull'architettura e il codice del sistema, con lo scopo di aggiungere alla tassonomia attuale istanze che sono specifiche e presenti esclusivamente nei sistemi intelligenza artificiale.