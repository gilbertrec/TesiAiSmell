\section{Motivazioni e Obiettivi} %\label{1sec:scopo}
L'emergere dell'intelligenza artificiale (AI) ha colpito in modo pervasivo le industrie software con la proliferazione di sistemi AI-based, integrando, quindi, le componenti AI nei sistemi software che operano nell'industria \cite{AnthesAI}.
Costruire, operare ed effettuare manutenzione dei sistemi AI però è differente dalle tradizionali metodologie di sviluppo e manutenzione di sistemi software. Infatti, in questi ultimi, il comportamento è dettato dal flusso di controllo del programma mentre, nei sistemi AI-based, le decisioni sono dettate dai dati. Questo nuovo punto di vista fa affiorare l'esigenza di esplorare le metodologie e le pratiche di Software Engineering per sviluppare, manutenere ed evolvere i sistemi AI. Questo è ciò che la comunità di ricerca di Software Engineering ha nominato \textbf{\textit{Software Engineering for Artificial Intelligence}} (SE4AI).

Le industrie devono affrontare molte sfide al fine di creare soluzioni AI a larga scala per il mercato. Fischer et al. \cite{fischer2020ai} hanno condotto uno studio per scoprire le maggiori sfide nello sviluppo di applicazioni AI.
In particolar modo, tra le più grandi sfide che un professionista deve affrontare figurano il garantire la qualità dei dati e del software.
La gestione della qualità, sia dei dati che dell'applicazione, è una procedura fondamentale per le applicazioni AI, in quanto una scarsa qualità può portare a gravi conseguenze.
Si consideri, tra i numerosi incidenti che coinvolgono l'AI, il caso di Elaine Herzberg, morta a causa di un errore di un'autovettura autonoma di Uber.
Il modello integrato nell'autovettura ha effettuato un errore di classificazione sul pedone che stava attraversando la strada, portando quindi l'autovettura a proseguire \cite{AiIncident}.
E questo, purtroppo, non è l’unico caso di sistema AI che ha causato ingenti danni (sia a persone che economici) a causa di negligenza nel controllo di qualità del modello.
È necessario quindi porre particolare attenzione agli aspetti di qualità dei sistemi AI e in particolare, ai problemi che possono causare un degrado della qualità.
La definizione di tecniche di validazione abilitate a prevenire il crollo della qualità rappresenta una delle più grandi sfide nel campo SE4AI. In questo contesto sono cruciali l'identificazione, la diagnosi e la gestione del debito tecnico, noto come technical debt. Il technical debt (TD) viene definito come un insieme di scelte di design subottimali o soluzioni implementative che possono influenzare negativamente i dati e la qualità del codice \cite{Cunningham1992td}.
%possono fornire alla comunità di data scientist e software engineer strumenti appropriati atti al controllo della qualità durante l'evoluzione dei sistemi AI 
I sistemi di Machine Learning hanno una notevole predisposizione nell'essere affetti da technical debt, poiché sono esposti sia alle tradizionali tipologie di technical debt, sia a \textbf{\textit{Artificial Intelligence Technical Debt} (AITD)}.
Questi ultimi rappresentano le tipologie di technical debt che pregiudicano la qualità dei sistemi AI-based, e risultano difficili da rilevare perché presentano diverse granularità, arrivando quindi a definire la rilevazione anche a livello di sistema. È fondamentale, quindi, rilevare e mitigare tempestivamente le problematiche derivanti dal TD, attraverso ciò che viene denominato \textit{technical debt management}. Sebbene sia urgente e necessario gestire il TD dei sistemi AI, allo stato attuale non esistono soluzioni sufficientemente efficaci a rilevare e mitigare queste criticità.

L'obiettivo di questo lavoro di tesi è quello di fornire un contributo cospicuo sia al settore accademico sia al settore industriale per la identificazione delle minacce alla qualità del software AI.
Il lavoro di tesi prevede quindi un'analisi preliminare sistematica della letteratura al fine di revisionare lo stato attuale dell'arte e ottenere un quadro generale sulle istanze che possono provocare AITD.
Successivamente, questo lavoro si pone di contribuire alla definizione delle istanze di AITD, attraverso l'investigazione tramite survey su larga scala di esperti e professionisti che hanno esperienza nella realizzazione e la manutenzione di sistemi e la gestione di una pipeline AI.
Infine le risposte sono analizzate al fine di ottenere per ogni specifico problema che può causare AITD la sua frequenza, la sua severità e il suo impatto, ed estrarre infine soluzioni volte all'identificazione, alla diagnosi e alla mitigazione di AITD e che, quindi, migliorino la qualità del software AI.

\section{Risultati}
I risultati hanno mostrato un'alta diffusione di diverse istanze di AI Technical Debt, dove le più diffuse sono relative ai dati, alla configurazione del sistema, al codice e all'architettura.
Concentrando lo studio successivamente sulle ultime due tipologie, è stata riscontrata un'alta severità e un alto impatto secondo la percezione degli sviluppatori AI.
Infine, i professionisti AI attualmente non dispongono dei mezzi necessari al fine di effettuare processi di identificazione e mitigazione delle minacce definite tramite tecniche automatizzate, ma conducono ispezioni e revisioni manuali.
\section{Struttura della tesi}

Il lavoro di tesi è organizzato nel modo seguente: nel capitolo due si forniscono i principi teorici su cui il lavoro è basato, al fine di facilitare la comprensione del contesto e gli l'importanza della gestione del Technical Debt nei sistemi AI. Nel terzo capitolo vengono illustrati i lavori presenti nello stato dell'arte che hanno contribuito alla definizione di una tassonomia preliminare. Nel quarto capitolo è introdotta la metodologia adoperata. Nella prima parte del capitolo è descritto il processo di revisione della letteratura sistematico e il processo di estrazione delle informazioni da essa.
Nella seconda parte, invece, viene illustrata la metodologia utilizzata all'investigazione della percezione degli sviluppatori AI.
In particolare, viene illustrato il processo di conduzione del questionario, della selezione dei partecipanti, e della progettazione dell'esperimento.
Infine, nel quinto capitolo di questo lavoro sono presentati i risultati delle analisi condotte sulla frequenza, la severità e il loro impatto.
Nel sesto capitolo sono fornite le conclusioni del lavoro svolto e delineati gli sviluppi futuri.