La diffusione dell'intelligenza artificiale (AI) nei prodotti industriali ha condotto alla creazione di sistemi in continua crescita, portando alla trasformazione di modelli sperimentali in prodotti commerciali.
L'esigenza di favorire l'evoluzione di questi sistemi ha portato all'introduzione delle pratiche di ingegneria del software mirate ai sistemi di intelligenza artificiale.
In particolare, la necessità di preservare la qualità dei modelli AI ha incrementato l'interesse della ricerca e dell'industria nel trovare metodologie utili a affrontare le più diffuse minacce alla qualità di questi sistemi.
In questo contesto, lo studio e le analisi di prevenzione delle tipologie di Technical Debt specifiche per AI (AI Technical Debt) rappresenta attualmente una delle più grandi sfide, in quanto la gestione di esse consente di identificare e mitigare severe problematiche che influenzano diversi aspetti di qualità dei sistemi.
Mentre l'importanza di gestire AI Technical Debt è emergente, l'attuale sforzo della ricerca e dell'industria non è sufficiente a fornire una ben definita tassonomia e catalogazione delle istanze che possono causare AI Technical Debt nei sistemi AI.
Il lavoro di tesi definito, quindi, fornisce un cospicuo contributo alla definizione delle istanze di AI Technical Debt, offrendo alla comunità accademica e industriale di poter analizzare la frequenza, la severità e l'impatto sugli aspetti di qualità delle possibili minacce che impediscono l'evoluzione dei moduli AI all'interno dei sistemi.
Di conseguenza, è stato condotto uno studio preliminare della letteratura presente nello stato dell'arte, al fine di estrarre le tipologie e le istanze di AI Technical Debt più diffuse.
Successivamente, il lavoro di tesi presenta l'investigazione del punto di vista di 54 professionisti di sistemi AI sulla frequenza, severità e impatto di istanze che possono causare AI Technical Debt sulla struttura del codice e sull'architettura del sistema.
I risultati mostrano un'alta diffusione e discussione in letteratura delle istanze di AI Technical Debt relative ai dati (\textit{data debt}), in particolare per le istanze che identificano dipendenze instabili tra i dati e le istanze che identificano informazioni e dati poco utilizzati dal modello.
I risultati estratti sia dallo stato dell'arte che dalla pratica evidenziano, inoltre, un'alta influenza delle istanze di AI Technical Debt relative al codice (\textit{code debt}) e all'architettura (\textit{architectural debt}).
In particolare, le istanze di \textit{Undeclared Consumers}, \textit{Pipeline Jungle} e \textit{Jumbled Model Architecture} sono state identificate essere tra le più severe e conseguire in un alto impatto nella qualità dei sistemi AI.
